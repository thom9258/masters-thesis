\documentclass[../main.tex]{subfiles}
\graphicspath{{\subfix{../src/}}}


\begin{document}
\section{Methodology \& Implementation}

\subsection{Anatomy}
\label{sec:anatomy}

The hand is an anatomically-complex appendage designed to facilitate a large amount of control in different usage scenarios.
The hand consists of 27 bones, 14 of these are called phalnages, and make up the 4 fingers and the thumb.
These bones, alongside a complex set of $\sim30$ muscles are able to perform $24$ Degree-of-Freedom (DoF) of rotational motion.
The individual finger consists of 3 bones called \gls{phalanges}, arranged linearly from the palm of the hand.
The 3 finger bones are called the proximal phalange, middle phalange and distal phalange.
The joints between the phalanges have 1 DoF and are able to do \gls{flexion/extension} movement, while the base of the finger have 2 DoF and are able to do \gls{flexion/extension} \& \gls{abduction/adduction} movement.
The anatomical muscle and joint structure of the hand/forearm can be seen in the figure \ref{fig:anatomy}.

\begin{figure}[h]
\begin{center}
\includegraphics[width=0.8\textwidth]{example-image-a}
\caption{A rendering of the anatomical muscle and joint strucutre of the lower-arm. The complexity of the muscles controlling the hand can be seen. Source \cite{???}}
\label{fig:anatomy}
\end{center}
\end{figure}

\subsection{Brief of used Software}
\label{sec:software}

In order to design a simulated prosthetic device that facilitates the requirements stated in section \ref{sec:prost_sim}, an effective simulation software needs to be chosen.
The software needs to be controllable from an external source, such as ROS, and because of this, software like Gazebo \cite{gazebo} or CoppeliaSim \cite{coppeliasim} is ideal.
Both simulation softwares allow for the creation of advanced dynamic-body simulations.
CoppeliaSim \cite{coppeliasim} was chosen due to its intuitive development workflow.

In order to record the joint movements of the hand, while simultaneously record the sEMG activity of the body, products that support hardware-synchonization needs to be used.
Products designed to capture sEMG data are widely spread.
As mentioned in the paper \cite{Zhaolong2021}, the Myo armband \cite{myo} or the Delsys Trigno package \cite{emgworks} could be used.
The Delsys Trigno \cite{emgworks} facilitates a hardware-synchonization recording mode out of the box, while the myo armband \cite{myo} does not.
Because of availability, and synchronization capabilities it is ideal to use the Delsys Trigno as the sEMG recording device.

Lastly, a product designed to capture the joint movements of the hand needs to be chosen.
A simple solution to hand tracking would be to use a software like MediaPipe \cite{mediapipe}, a python based landmark tracker that can track the joint angles of the hand using Machine Learning from a video feed.
An alternative is mentioned in the paper \cite{Zhaolong2021}, where a glove containing flex sensors could be used.
This approach would provide relieable joint angle data without relying on a camera setup.
Lastly, a capture glove could be fitted with 3D markers, detectable from a high-accuracy and high-framerate system such as the Motive motion capture setup \cite{optitrack} created by OptiTrack.
It was chosen to use the Motive setup due to availablility and the additional benefit of having a hardware-synchnonization feature, that can be combined with the Delsys Trigno \cite{emgworks} setup.


\subsection{Dataset Creation}
\label{sec:dataset}

In order to train a simulated hand prosthetic, a sofisticated dataset containing the measured relation between muscle activity and the finger placements is needed.
The paper \cite{KeunTaeKim2021} proposes the SHAP procedure \cite{shap} as its main method of dataset creation.
An alternative would be the Sollerman test \cite{sollerman} as both procedures are used in assessing the function and moveability of the human hand.
By analyzing the most important gripping motions in both tests, it should be possible to denote a suitable set of grips that the dataset should contain.
The sollerman grip types are ranked based on their usage percentage in activities of daily living.
The most used grip types according to sollerman are the \textit{Pulp pinch}, \textit{Lateral pinch}, \textit{Five-Finger pinch} \& \textit{Diagonal Volar grip (Power grip)}, see \cite{sollerman} for further details.
SHAP also proposes a set of grip types that are used in day-to-day tasks, these are \textit{Spherical grip}, \textit{Tripod pinch}, \textit{Power grip} \& \textit{Lateral pinch}.

In order to create a dataset mimics the movements of day-to-day tasks, the most important grips from \cite{sollerman} \& \cite{shap} has been chosen.
The grip types that needs to be part of the dataset and their usage descriptions can be seen in table \ref{tab:grips}.

\begin{table}[h]
\begin{center}
\begin{tabular}{ |l|l| } 
 \hline
 Grip Type & Finger Usage Description \\ 
 \hline
 Pulp pinch & Between thumb, index and middle finger \\ 
 Lateral pinch & Between thumb \& side of index finger \\ 
 Five-Finger pinch & Between thumb, and all four fingers \\ 
 Power grip & Between thumb, and all four fingers with contact to palm \\ 
 \hline
\end{tabular}
\caption{The most used hand grips in day-to-day tasks based on \cite{sollerman} \& \cite{shap}.}
\label{tab:grips}
\end{center}
\end{table}

A set of consise grip types has been chosen as an alternative to a larger set of general grips.
This is done as a basis of creating a specialized dataset that would be easier to train and work with as a proof of concept.
%The chosen grips are chosen as a basis for the creation of a dataset. creating 


\subsubsection{Existing datasets}

In addition to creating a dataset for this thesis, it would be intresting to compare with an existing state-of-the-art dataset.
The paper \cite{jarque2019} proposes the use of their dataset \cite{kinmusdataset}.
The dataset contains a very large set of recordings, consisting of precise anatomical angles of the hand, with the assosicated muscle activity of the forearm.
The dataset consists of $572$ recordings from $22$ subjects, in both reaching, gripping \& releasing actions.

Another dataset is explained in the paper \cite{ashirbad2022}.
As an alternative of using finger joint angles as the ground truth data needed to train a network on sEMG data, the dataset uses a set of $16$ gesture classes for a classification algorithm.
The paper explains that the sEMG data in the dataset has been subject to a pre-processing step using a $10$ to $500Hz$ bandpass buttersworth filter, and in order to remove powerline noise a $60Hz$ notch filter was used.
The dataset was created on $43$ participants.


%The recording of the dataset is done using the software explained in section \ref{sec:software}, namely Motive \cite{motive} \& EMGworks \cite{emgworks}.
%TODO: Present more formally and mention company names.

\subsubsection{Sensor Locations etc.}

The muscle recording sensors are located along the muscles of the forearm, the exact positions can be seen in figure \ref{fig:musclesensors}.

\begin{figure}[h]
\begin{center}
\includegraphics[width=0.8\textwidth]{example-image-a}
\caption{Example figure text}
\label{fig:musclesensors}
\end{center}
\end{figure}

%\subsubsection{Trial/motion overview}

\subsubsection{Motion Capture Glove}

The motive capture system created by OptiTrack is a set of 8 high-quality cameras mounted to cover a target capture area.
%TODO: how many cameras were there?
The capture system detects flouresent 3D markers in the given capture area, with the purpose of triangulating the markers and effectively calculate 3D poses for the markers in the scene down to an effective accuracy of $\pm 0.2mm$ \cite{motive}.
In order to get precise recordings of the motion of the hand and fingers, flourecent 3D markers were placed on a glove.
The pattern of the marker positions were closen in order to calculate the angles of the individual finger bones.
The markers are used in sets of 3, this allows for the calculation of the triangle angles in 3D space.
The positions of the 3D markers on the recorder glove can be seen in figure \ref{fig:glove}.
An example of flexion with the capture glove can be seen in figure \ref{fig:glove_flex}.


\begin{center}
\begin{figure}[h]
\includegraphics[width=0.8\textwidth]{example-image-a}
\caption{The positions of 3D markers on the capture glove, designed to be detectable using the  Motive Capture software \cite{motive}.}
\label{fig:glove}
\end{figure}
\end{center}

\begin{center}
\begin{figure}[h]
\includegraphics[width=0.8\textwidth]{example-image-a}
\caption{Flexion of the index finger and its transformation of the 3D markers on the capture glove.}
\label{fig:glove_flex}
\end{figure}
\end{center}

\textbf{Problems with Design of the Capture Glove and its Software}

During testing several problems with the capture glove design became apparent.
The Motive sofware used to generate the 3D poses of the markers uses an optimization algorithm that effectively tries to determine if the markers seen in 1 camera is the same as the markers in the other cameras.
This optimization technique is prone to specular reflection in the scene caused by shiny surfaces redirecting light in the target area.
Furthermore, the optimization algorithm seemed during recording to be optimizing 3D markers adjacent to each other into the same 3D pose.
This problem occoured when the markers at the finger tips of the glove came in contact with each other causing data to be lost.
On top of aggressive optimization, the markers also had a tendency to dissapear if less than 3 cameras were able to detect it, causing problems with the resulting continurity of the 3D poses. 

%TODO: Muscles of the hand: https://www.assh.org/handcare/safety/muscles
%TODO: Bones of the hand: https://www.bidneedham.org/departments/orthopaedics/hand-program/anatomy-hand-and-wrist

\subsection{Design of a Simulated Prosthetic Hand}
\label{sec:prost_sim}

In order to design a state-of-the-art simulated prosthetic hand, a number of anatomical design choises needs to be considered.
This thesis tries to create the most anatomically-correct hand simulation available, this will hopefully have a number of positive effects on prosthetics research.
By having access to an advanced simulation, it would in turn be able to test and visualize more advanced movement controllers that can facilitate more DoF than current commercial prosthetics. 
By creating an anatomically correct prosthetic hand simulation, it is hoped that prosthetics users can have more advanced rehabilitation, and learn to have more natual control of their prosthetics. This would create a more natual usage experience, and decrease the percentage of users that reject the usage of their prosthetic alltogether.

A set of requirements The simulated anatomically correct hand should be determined in order to create a state-of-the-art prosthetics simulation.

The simulated prosthetic should:

\begin{enumerate}
\item Facilitate the same DoF as an anatomically correct hand.
\item Have porpotions that closely resemble that of an anatomically correct hand.
\item Be simulated and be controllable in a commonly used robotics software to increase accesability for researchers.
\end{enumerate}

\subsubsection{Simulated Hand Articulation design}

%TODO: see Yuki2023 (page 11) for how to properly show the dimensions of my hand!
In order to translate the biology and anatomy of a real hand explained in \ref{sec:anatomy} into an robotics simulation, we start by denoting the relative lengths of the wrist bones and phalnages by refrence, as can be seen in figure \ref{fig:handref}.

%TODO: IMAGE HERE OF HAND OR XRAY
\begin{figure}[h]
\begin{center}
\includegraphics[width=0.8\textwidth]{example-image-a}
\caption{Example figure text}
\label{fig:handref}
\end{center}
\end{figure}

The porpotions of the refrence is used to denote the bone lengths for the model.
The model is implemented in Coppeliasim \ref{???}, The model is created in a hierachy, the bones are created with cylinders and the joints are created using 1 DoF Revolute joints.
% Ref coppeliasim
As specified in section \ref{sec:anatomy}, some joints of the human hand facilitates 2 DoF of rotation.
This is needed in order for the wrist and finger base joints to be able to do \gls{abduction/adduction}.
To simulate this, two 1 DoF revolute joints were placed in series, thus allowing 2 DoF.


% \section{Implementation}

Based on the literature review in Section \label{sec:literature}, it becomes apparent that there exists a standard pipeline for translating sEMG recordings into predicted motorcontrol for a prosthetic.

\subsection{sEMG Data Processing}

Raw sEMG data contains a lot of unwanted noise as explained in section \ref{sec:noise}.
Some of the noise can be can be removed through signal processing using filters.
State-of-the-Art papers use a lot of different methods to reduce noise in sEMG data.
The paper \cite{multdof} proposes the use of a $50Hz$ Notch filter, while papers \cite{graspintent} \& \cite{ashirbad2022} respectively chooses a $20Hz$ cutoff Buttersworth filter \& a $10-500Hz$ band-pass Buttersworth filter.
Due to me many different filtering types used in state-of-the-art, it was determined that the use of a filter depends on the specific sEMG data, and that the best filter for this thesis is to be determined through testing.

\subsection{Network Design}
\subsection{Software Hand Design}

\end{document}
