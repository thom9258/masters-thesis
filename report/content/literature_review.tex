\documentclass[../main.tex]{subfiles}
\graphicspath{{\subfix{../src/}}}


\begin{document}

\section{Literature Review}

As sensors capability for for biological sensoring increases in state-of-the-art prosthetics development and research, there becomes a larger need for translating sensor data into usable input data for prosthetics controllers.
A lot of reserach has been done in this area, this research elaborates on different Machine learning or AI-based methods of understanding muscle-based sensor data.
The pipeline for converting EMG sensor data to usable input data often contains a pre-processing step, where data is de-noisified, and cleaned of potential errors.
The pre-processing step can also contain feature extraction such as ...
%TODO: Mention some example stuff like extracting features of raw data, i think RMS, and other spaces.
The pre-processing step is then followed by a processing step, this step encompasses the use of a Machine-learning algorithm or a Neural Network, designed to either Classify a grip type, or Regress the angle of the joints.
After Classification or Regression a post-processing step can be added where the actual kinematic data is created, and used as input to the prosthetic controller. 
Popular methods of processing EMG signals will be researched, and elaborated upon, with the aim of identifying robust, effective and implementable methodologies.
% that can be tested in the context of the thesis, and the dataset i create...
%TODO: Can i do something like this?


\end{document}
