\documentclass[../main.tex]{subfiles}
\graphicspath{{\subfix{../src/}}}


\begin{document}
\section*{Abstract}

% Introduction of problem
The human upper-limbs are some of the most important appendages of the human body.
The upper limbs of a human allows for interaction and manipulation of the environment, and acts as a crucial part of a functioning human body.
% Why we need to solve this problem
The loss of an upper limb, either congenital or traumatic severely reduces a person's ability to interact with- and perform simple day-to-day tasks in the environment required for basic human living.
% Introduction to existing solutions and their flaws
This is a major problem for the amputee, and can only be partially solved by rehabilitation using a crude prosthetic device, designed to provide a interface between a mechanical or robotic appendage, and the remaining appendage section of the amputee.
These prosthetic devices often only allow for a small degree of control through a limited interface at the attachment point of the prosthetic.
This often causes the amputee to neglect the rehabilitation and usage of the prosthetic, by choosing to reserve it for limited use, or reject it all together.
Amputees not choosing to make use of a prosthetic device, often cause overuse of their remaining limbs.
% This thesis researches.... Intro to my work
This thesis aims to decrease the gap between existing prosthetic devices and their real-life counterparts, with the purpose of increasing control and amount of interaction for the amputee, and by doing so, decreasing the amount of amputees choosing to reject their prosthetic device. 
This thesis researches different control methods of a prosthetic device, with a focus on processing and interpreting sEMG data recorded from the upper appendage and the upper body.
% This thesis proposes... outcome of my work
This thesis researches and compares different methods of Machine Learning and Artificial Intelligence in order to convert muscle recordings into a controllable output for a prosthetic device.
This thesis proposes the use of feature-based Linear Discriminant Analysis for intent classification, and the usage of a 4-layer Recurrent Neural Network to predict the joint movements of the hand.
This thesis explores creation of a training dataset using the Motive motion capture system, in cooperation with the Delsys Trigno muscle recording system, and a custom made motion capture recording glove for the joint movements of the hand.
Lastly, this thesis proposes an anatomically correct simulated prosthetic hand, designed to visualize and test control methods of a prosthetic device.

%TODO: Should i have results in here?
%These methods provide 

%TODO: Some kind of ending sentence to round it out.

\end{document}
