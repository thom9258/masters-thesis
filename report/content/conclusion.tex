\documentclass[../main.tex]{subfiles}
\graphicspath{{\subfix{../src/}}}


\begin{document}
\section{Conclusion}

This thesis is developed with the purose of researching \& designing a state-of-the-art sEMG-based HMI for a simulated prosthetic hand.
During the development of this thesis, a controllable prosthetic hand was developed in the simulation tool CoppeliaSim \cite{coppeliasim}, that is able to be controlled by the commonly used robotics communication software ROS2 \cite{ros2}.
The prosthetic hand is designed to have anatomically correct porpotions, with anatomically correct joints that allows the simulation to act and be controlled with similar movements as a real hand. 
The simulation is created with the purpose of providing an easy-to-use, freely available, advanced prosthetics simulation for others to use, modify and develop with. 
Another purpose of this thesis is to research and test state-of-the-art HMI techniques in the field of AI and ML, in order to compare and test different methods of converting sEMG based muscle activity into a useable controller scheme for a prosthetics device.
This thesis tests two different areas of HMI interfaces, ML based lower-arm intent classification and AI based finger movement regression.
Different models in each area are tested and compared, based on reccomendations of state-of-the-art
literature, in order to determine what models are ideal for the control of a prosthetic hand.
Several different ML models were trained and compared in section \ref{sec:ML}.
All models were trained on feature-space converted sEMG data.
Based on the tests and comparison of the classification accuracy table \ref{tab:classification_comparison}, it can be seen that the Linear Discriminant Analysis (LDA) performs best in average accuracy of $0.61\%$ and interquantile range of $0.679\%$.
It can also be seen, and that QDA performs best in the interquantile median with a value of $0.803\%$.
From this, the LDA machine learning algorithm is proposed to be best at intent classification of the lower-arm.
As an alternative to intent classification, different AI models designed to do regression of finger movements were trained and compared in section \ref{sec:AI}.
Based on the regression comparison in figure \ref{fig:regression_comparison}, it can be seen that Recurrent methods perform best compared to window based methods.
This can be seen when comparing the average error in 50 tests, the CNN network has an average error of $0.17$ Radians per test, where the RNN network has a low average error of $0.01$ Radians.


% in debth of classification
% in debth of regression
% in debth of dataset creation proposal
% with cleaning of data 
% Chosen hardware stuff explained 
% Chosen hardware stuff explained 


\subsection{Future Work}

\end{document}
