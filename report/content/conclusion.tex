\documentclass[../main.tex]{subfiles}
\graphicspath{{\subfix{../src/}}}


\begin{document}
\section{Conclusion}

This thesis is developed with the purpose of researching \& designing a state-of-the-art sEMG-based HMI for a simulated prosthetic hand.
% Chosen hardware stuff explained 
During the development of this thesis, a controllable prosthetic hand was developed in the simulation tool CoppeliaSim \cite{coppeliasim}, that is able to be controlled by the commonly used robotics communication software ROS2 \cite{ros2}.
The prosthetic hand is designed to have anatomically correct proportions, with anatomically correct joints that allows the simulation to act and be controlled with similar movements as a real hand. 
The simulation is created with the purpose of providing an easy-to-use, freely available, advanced prosthesis simulation for others to use, modify and develop with. 
Another purpose of this thesis is to research and test state-of-the-art HMI techniques in the field of AI and ML, in order to compare and test different methods of converting sEMG based muscle activity into a usable controller scheme for a prosthetic device.
This thesis tests two different areas of HMI interfaces, ML based lower-arm intent classification and AI based finger movement regression.
Different models in each area are tested and compared, based on recommendations of state-of-the-art
literature, in order to determine what models are ideal for the control of a prosthetic hand.

% in debth of classification
Several different ML models were trained and compared in section \ref{sec:machine-learning}.
All models were trained on feature-space converted sEMG data.
Based on the tests and comparison of the classification accuracy table \ref{tab:classification_comparison}, it can be seen that the Linear Discriminant Analysis (LDA) performs best with an accuracy of $61\%$ and inter-quantile range of $0.679$.
It can also be seen, and that QDA performs best in the inter-quantile median with a value of $0.803$, but has the largest inter-quantile range.
From this, the LDA machine learning algorithm is proposed to be best at intent classification of the lower-arm.
% in debth of regression
As an alternative to intent classification, different AI models designed to do regression of finger movements were trained and compared in section \ref{sec:regression}.
Based on the regression comparison in figure \ref{fig:regression_comp}, it can be seen that Recurrent methods perform best compared to window based methods.
This can be seen when comparing the average error in 50 tests, the CNN network has an average error of $0.17$ Radians per test, where the RNN network has a low average error of $0.01$ Radians.
% in debth of dataset creation proposal

The designed models are trained and tested using a freely available dataset \cite{kinmusdataset}.
This was done due to problems with the chosen Motive Motion Capture setup, these problems were reduced by the implementation of a rendering and cleaning software explained in section \ref{sec:motivecleaning}.
Using this implemented software, it is possible to fixing wrong labels and removing labels all together, compared to the building functionality in Motive.
%This software makes fixing wrong labels and removing labels all together possible, compared to the building functionality in Motive.
With this implemented software, it was possible to clean a small subset of recorded data, but it was not possible to create a full dataset of multiple people.
% with cleaning of data 

%TODO: Emphasis on what the result of my thesis is! What specifically am i proposing to use in further research?

%During the development of this thesis, a 


% TODO: Take it further and go into detail on my goals and what worked and what didnt

\newpage
\subsection{Future Work}

Due to the problems encountered with the chosen software in this thesis, some parts of the motivation goals in section \ref{sec:goals} were not possible to explore.
Had it been easier to create a custom dataset, preferably with recordings from multiple people, it would have been possible to explore further combination of the created prosthetic and the trained models.

Further testing, of the dynamic control of a trained model in a simulated environment could also have been explored.
This could be done in real-time, where a trained model could be used to convert live sEMG recordings into movements for the simulated prosthetic, allowing the test person to manipulate the simulated environment in real time.
The work done in this thesis mainly explores two different areas of prosthetic controllers, classification based and regression based.
Had a large dataset created specifically for the purpose of this thesis been created, it would be ideal to test the applicability of combining methods and by doing so, explore possible controller schemes that combines the individual movement of regression models and the robustness of intent classification.  
Different network types could also have been tested, with further emphasis of translating the network models into robust controller information.
This could have been done with a focus on testing different methods of connecting multiple network types into a single controller, or describing a set of base movement behaviors that a model can be integrated with to create movements that are dynamic, but also predictable to the user.  
This could have been further expanded by experimenting with automatic grip methods, such as using sensor feed back to optimize gripping. 

Lastly, it is apparent that using sEMG sensors as the interface between an amputee and their prosthetic device has limitations such as being highly dependent on the severity of the loss, and that prosthetic controllers relying primarily on sEMG data is unable to recreate all of the dynamic control of the real hand.
Future research could be focused on exploring alternative methods, either by improving \gls{EEG} recording or exploring methods of reading neural activity directly from the upper-arm.

% TODO: Go back to problem and create some pointers to what is applicable based on my work

\end{document}
