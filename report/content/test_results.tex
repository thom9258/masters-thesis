\documentclass[../main.tex]{subfiles}
\graphicspath{{\subfix{../src/}}}


\begin{document}
\section{Tests \& Results}

\subsection{1. (Machine Learning Algorithms)}
%TODO: Determine what algorithms i want here

\subsubsection{Types of Features for extraction}
\textbf{Method}\\
\textbf{Test}\\
\textbf{Results}\\

\subsection{2. Windowed MLP/CNN}

\subsubsection{Pre-Processing Filters}
\textbf{Method}\\
\textbf{Test}\\
\textbf{Results}\\

\subsubsection{Window Size}
\textbf{Method}\\
\textbf{Test}\\
\textbf{Results}\\

\subsubsection{Training Parameters}
\textbf{Method}\\
\textbf{Test}\\
\textbf{Results}\\

\subsubsection{Types of Features for extraction}
\textbf{Method}\\
\textbf{Test}\\
\textbf{Results}\\

\subsection{3. Recurrent Neural Networks}

\subsubsection{Training Parameters}
\textbf{Method}\\
\textbf{Test}\\
\textbf{Results}\\


\subsection{Useability of the Simulated Prosthetic Hand}

Existing prosthetic hand simulations are not possible to get access to, as explained in section \ref{sec:hand_alternatives}, they are either unavailable through deprecation or pay-to-acces by buying a real prosthetic.
Due to non-availability of prosthetic hand simulations, it was chosen that through this thesis, a open source prosthetic hand needed to be created for testing of the algorithms developed as part of this project.
The prosthetic hand created for this thesis can be seen in section \ref{sec:prost_sim}.
The prosthetic was designed to be as anatomically similar as a real hand as possible.
This was done in order to accurately simulate the movement of a real hand.
%TODO: Is this intro good?

\subsubsection{Anatomical Assessment and Maneuverability}

The useability and moveability of the simulated prosthetic hand needs to be assessed.

\textbf{Method}\\
As an initial test, it is reasonable to test if the hand is able to acheive the end poses of the grip types from section \ref{sec:dataset} table \ref{tab:grips}. 

\textbf{Test \& Results}\\

The hand was manually posed into all 4 grip types, and visually compared, as can be seen in figure  \ref{fig:hand_pose_test}.

%TODO: Needs to be a figure of figures!
\begin{figure}[h]
\begin{center}
\includegraphics[width=0.8\textwidth]{example-image-a}
\caption{Example figure text}
\label{fig:hand_pose_test}
\end{center}
\end{figure}

Based on a visual comparison of figure \ref{fig:hand_pose_test}, it can be concluded that the prosthetic hand is able to be posed similarly to the real-life refrence.
It can be seen how the prosthetic is porpotionally correct and thus also allows for accurate posing when posed manually.

\subsubsection{Posing based on Network Output}

As a result of testing different network types and their applicability for creating suitable motorcontrol output for a prosthetic hand, it would be ideal to test the network output on the prosthetic simulation.
For full video see appendix ??.
%TODO: Insert appendix here

\textbf{Method}\\

\textbf{Test \& Results}\\



%\subsection{Method 1: Windowed CNN}
%\subsection{Test 1: Windowed CNN}
%\subsection{Results 1: Windowed CNN}
%\subsection{Method 2: RNN}
%\subsection{Test 2: RNN}
%\subsection{Results 2: RNN}


%TODO: see Zhaolong2021 (page 10) for example of how to compare MSE for different neuron amounts, even for multiple subjects!

\end{document}
