\documentclass[../main.tex]{subfiles}
\graphicspath{{\subfix{../src/}}}


\begin{document}
\section{Problem Specification}

There is a large need for new technology that improves the effectiveness and ergonomics of human hand prosthetics.

Current state-of-the-art products on the market exhibits a severe reduction of controllable Degree of Freedom (Dof) compared to their biological counterparts.
These products often rely on simple, grasp control based on 2 or more \gls{sEMG} interfaces, to classify ``open/close'' signals for the control scheme.
The prosthetics user is then manually required to change grip control-scheme, creating very crude control dynamics that is very different from biological hand-control.
A in-debth explanation of ``open/close'' control can be seen in section \ref{???}
%TODO: This sounds weird, integrate a simple controller therory here, and why it is so crude, then refrence it in state-of-the-art
%TODO: Create refrence
This is a great pitfall in the field of Research and Creation of prosthetics, as unsatisfactory function of prosthetics lead to amputees, that exhibit a great deal of stress douring the rehabilitation process.

%TODO: Refrence for this!
This can cause the patient to repel the rehabilitation process and the prosthetic all-together.
The repelling of the prostetic increase in the cases of the most severe cases of amputation, where the largest amount of control muscles are lost.
These amputations are often located further up the limbs, where the loss of mobility and controllability are greatest.
The amount of muscles leftover from amputation also dicates the type of prosthetic a patient is able to recieve.
%TODO: Refrence for this!
Patients of lower-arm amputation has less control over their prosthetic than patients of hand amputation, due to the loss of the muscles in the lower-arm.
The loss of control increases as the amputation severity increases, and this is a problem in prosthetics design because it is impossible to create a standardized controller that suits most patient's needs.

State-of-the-art commercial prosthetics further decrease the controllable DoF in order to increase robustness of the control experience, this is further elaborated upon in \ref{sec:stateoftheart}.
%TODO: Create refrence for state-of-the-art

\subsection{Motivation}

The main goal of this thesis is to provide a meaningful contribution to the world of prosthetics design and control.
In order to confine the workload done in this thesis, a set of development goals has been made:

\begin{enumerate}
\item Create a software-based, biology-inspired, anatomically realistic simulation of a humanoid lower-arm/hand that is able to imitate the movements of the humanoid limb.
\item Make the prosthetics simulation controllable from a widely-used robotics-software.
\item Design a sEMG muscle pre-prossesing pipeline for a prosthetics controller. 
\item Design a state-of-the-art prosthetics controller based on AI, to control a simulated prosthetics device.
\item Create a custom dataset to train AI based controllers for prosthetics.
\item Test and Validate the created prosthetics controller against state-of-the-art methods.
\end{enumerate}
%TODO: This needs


\end{document}
