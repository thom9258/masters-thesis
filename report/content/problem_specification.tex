\documentclass[../main.tex]{subfiles}
\graphicspath{{\subfix{../src/}}}


\begin{document}
\section{Problem Specification}

There is a large need for new technology that improves the effectiveness and ergonomics of human hand prosthetics.
Current state-of-the-art products on the market exhibits a severe reduction of controllable Degree of Freedom (Dof) compared to their biological counterparts.
This is a great pitfall in the field of Research and Creation of prosthetics, as unsatisfactory function of prosthetics lead to amputees, that exhibit a great deal of stress douring the rehabilitation process.
%TODO: Refrence for this!
This can cause the patient to repel the rehabilitation process and the prosthetic all-together.
The repelling of the prostetic increase in the cases of the most severe cases of amputation, where the largest amount of control muscles are lost.
These amputations are often located further up the limbs, where the loss of mobility and controllability are greatest.
The amount of muscles leftover from amputation also dicates the type of prosthetic a patient is able to recieve.
%TODO: Refrence for this!
Patients of lower-arm amputation has less control over their prosthetic than patients of hand amputation, due to the loss of the muscles in the lower-arm.



\end{document}
