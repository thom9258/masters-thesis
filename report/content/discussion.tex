\documentclass[../main.tex]{subfiles}
\graphicspath{{\subfix{../src/}}}


\begin{document}
\section{Discussion}

The main purpose of this thesis is to research and experiment with state-of-the-art sEMG-based human-machine-interfaces for development of prosthetic devices.
The experimentation is mainly done with the focus of identifying and modeling the intent \& movement of the fingers.
% Mention gazebo and copelliasim problems
During the development of this thesis, several problems occurred with the chaise of simulation tools.
The initial chosen simulation software was the Gazebo \cite{gazebo} due to its close integration with ROS2 \cite{ros2}, and its use in modern robotics simulation tasks.
After a lot of long-term problems and complications caused by the chosen software, a decision to switch to CoppeliaSim \cite{coppeliasim} was made.
The change to CoppeliaSim solved a lot of the problems caused by Gazebo, the setup of the simulation and development of the kinematics became much clearer.
Furthermore, setting up the controller of the prosthetic could be done using the build-in ROS2 communication.
%TODO: Problems with motive
%      Data unreliability
%      having to make my own methods to clean data
%      Unable to create a proper dataset with own methods
%      Unable to have multiple people in the dataset due to extensive clearning
Another problem had a large impact on the development of this thesis.
In order to capture the movements of the fingers for the dataset, the product Motive Motion Capture Setup \cite{motive} was chosen, due to recommendation, availability and its use in state-of-the-art industry motion capture.
The Motive setup consists of a camera rig designed to locate 3D markers in the scene.
As a recording target, a glove was fitted with 3D markers.
This was done in order to record the kinematic movements of the hand, a glove is fitted with 3D markers at the joints of the fingers.
%so that it would be possible to calculate the angles between the 3D markers.
The main problem is that the setup is designed to capture large objects such as full-body tracking suits and larger static objects.
Because of this, the 3D optimization algorithms used by the cameras has a tendency to remove markers that are close. 
This would cause the 3D markers of adjacent fingers to be combined into a single marker.
The only way to solve this is to reduce the amount of markers on the glove.
Furthermore, the 3D marker labeling is unreliable.
The markers are shifting in labels if they disappear for a few frames at a time.
This problem scaled with the length of the recording session, a 15 second recording of 7 markers would be recorded as $100+$ labels in the resulting tracker output.
In addition to the tracking being unreliable, the Motive software is unable to provide an effective label correction tool.
Label rendering \& correction tools is developed as part of this thesis, and is used to clean up a subset of the recorded dataset.
Due to the extensive work needed to record and clean the movements of the hand, it became impossible to create a dataset large enough to train models on, and to create a general purpose dataset with recordings from multiple people. 
The time-consuming process of creating a simulated prosthetic and recording a dataset resulted in some of the goals of this thesis as explained in section \ref{sec:goals} to not be fulfilled.
It was not possible to fully test the simulated prosthetic device with real-time predictions from a model. 
% if i had the time: compared kinematic output with control of the sim hand.
Furthermore, it was not possible to fully test different pre-processing methods and their differences when used for training a model.

%TODO: Explain that actual angle calculation was not done for recorded markers.

\textbf{Classification Tests}\\
% It would have been nice to test different types of feature extraction methods for classification.
Multiple different machine learning methods is tested in section  \ref{sec:machine-learning}
The data is converted to a feature space set consisting of zero-crossing and MSE, and is then used as input to a machine learning algorithm to create a lower-arm intent classifier.
The accuracy of the resulting classifiers could have been improved, if more types of feature extractions had been tested.
Furthermore, different types of machine learning classifiers and network based classifiers could have been tested in order to improve classification accuracy.

\textbf{Regression Tests}\\
Different regression network models were trained to do regression of finger joint angles. These tests can be seen in section \ref{sec:regression}.
Window based regression network methods were compared to recurrent network based methods.
The window based regression has a large error, and it is unclear what the exact cause is.
The large error could be result of using an incorrect size of window, or the result of the training data not being optimal for training.
The error could also be the result of the training parameters used, and it might have been solved by extensive parameter tuning.
The use of a recurrent network proved to have very accurate results for regression of finger angles.

\textbf{Hand Simulation Tests}\\
The simulated hand created in CoppeliaSim \cite{coppeliasim} is tested in section \ref{sec:simtest}.
The hand is designed to be anatomically proportional to a real hand, with the same amount of degree of freedom as its real counterpart.
The tests were created to assess the simulated prosthetic hand's ability to be manipulated and its ability to perform anatomically correct poses.
The hand was manipulated to the 4 most used day-to-day grip types with great precision by a visual assessment.
The main problem with the hand simulation is that it was not possible to test its performance using the output of a regression model as a controller.
This was not possible due to the created dataset.



%TODO: Tests not being too good, explain its due to dataset problems,
% Explain more time could have been used on methods fine tuning and experimentation, if there had been no problems with dataset creation.

%TODO: CNN network seems to approximate the median of all the ground truth regression angles.

% Results are liable to parameter tuning

% if i had the time: compared kinematic output with control of the sim hand.

\end{document}
