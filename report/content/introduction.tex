\documentclass[../main.tex]{subfiles}
\graphicspath{{\subfix{../src/}}}


\begin{document}

\section{Introduction}

The human hand is one of the most important factors of the human identity.
The hand allows a person  to perform complex muscolatory combinations to interact with the surrounding world, express complex emotions during speech, and aid in defining a person's individuality and personalty \cite{???}.
% TODO: cite "Grasping the Importance of Our Hands"
The hands are controlled by a complex combination on precise muscles designed to perform gentle, precise control of the fingers.
This allows a person to grasp objects in many different ways, perform complex tasks such as writing, playing musical instruments, or even constructing a house.
The hand also acts like a sensory device allowing us to perform precise observations through feeling and touch.
This allows a person to understand the environment without seeing it, the hand is able to sensor heat/cold, create complex understanding of geometries and texture through touch and manipulation.

Missing limbs, either \gls{congential} or \gls{traumatic} amputation severely reduces a person's ability to interact with- / understand the world, express themselves and perform simple day-to-day tasks.
In order to alleviate some of the drawbacks of missing a limb, amputees are often able to aquire a prosthetic replacement of their lost limb.
The aquired prosthetic tries to imitate the movements of the lost limb, through muscle-activated interfaces, that is then used to control the movements of the prosthetic. 
In the case of hand prosthetics, the prosthetic allows the user to perform simple, day-to-day tasks, and is able to alleviate some of the stress caused to the non-amputated hand through overusage.
%TODO: Refrence here
This thesis aims to summarize, and elaborate on current state-of-the-art research and products in the field of prostetics devices, and products the control of prosthetics and the existing limitations of these state-of-the-art products. 

This thesis aims to contribute to the world of prosthetics control, by researching effective methods of collecting sensory data from the lower-/upper-arm, and by doing so, creating an state-of-the-art Artificial Intelligence (AI) based controller, that is able to imitate the intent and movements or a real hand.
And by doing so, by improving sEMG controller design to increase functionality and the controllable DoF of the prosthetic, to provide a more true-to-life experience to the prosthetics user, and thus reduce the amount of patients that disregard prosthetics.

This thesis also aims to explore efficient methods of designing a network to identify lower-/upper-arm muscolatory intent, with the purpose of controlling a simulated prosthetics device, and by doing so, increase the controllable Degree-of-Freedom for the prostetics user.


\end{document}
