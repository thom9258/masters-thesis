\documentclass[../main.tex]{subfiles}
\graphicspath{{\subfix{../src/}}}


\begin{document}

\section{Introduction}

The human hand is one of the most important factors of the human identity.
The hand allows a person  to perform complex muscolatory combinations to interact with the surrounding world, express complex emotions during speech, and aid in defining a person's individuality and personalty \cite{???}.
% TODO: cite "Grasping the Importance of Our Hands"
The hands are controlled by a complex combination on precise muscles designed to perform gentle, precise control of the fingers.
This allows a person to grasp objects in many different ways, perform complex tasks such as writing, playing musical instruments, or even constructing a house.
The hand also acts like a sensory device allowing us to perform precise observations through feeling and touch.
This allows a person to understand the environment without seeing it, the hand is able to sensor heat/cold, create complex understanding of geometries and texture through touch and manipulation.

Missing limbs, either \gls{congential} or through amputation severely reduces a person's ability to interact with- / understand the world, express themselves and perform simple day-to-day tasks.
In order to alleviate some of the drawbacks of missing a limb, amputees are often able to receive a prosthetic replacement of their lost limb.

This thesis aims to summarize, and elaborate on current state-of-the-art research in the field of prostetics devices, the control of prosthetics and the existing limitations of these state-of-the-art products. 

This thesis aims to contribute to the world of prosthetics control, by researching effective methods of collecting sensory data from the lower-/upper-arm, and by doing so, creating an state-of-the-art Artificial Intelligence (AI) based controller, that is able to imitate the intent and movements or a real hand.


This thesis also explores efficient methods of 

\end{document}
