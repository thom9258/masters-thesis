\documentclass[../main.tex]{subfiles}
\graphicspath{{\subfix{../src/}}}

\begin{document}

\section{Introduction}

%HANDS ARE IMPORTANT
The human hand is one of the most important factors of the human identity.
The hand allows a person  to perform complex muscolatory combinations to interact with the surrounding world, express complex emotions during speech, and aid in defining a person's individuality and personalty \cite{Douglashands}.
The hands are controlled by a complex combination on precise muscles designed to perform gentle, precise control of the fingers.
This allows a person to grasp objects in many different ways, perform complex tasks such as writing, playing musical instruments, or even constructing a house.
The hand also acts like a sensory device allowing us to perform precise observations through feeling and touch.
This allows a person to understand the environment without seeing it, the hand is able to sensor heat/cold, create complex understanding of geometries and texture through touch and manipulation.
% AMPUTEE PROBLEMS
Missing limbs, either \gls{congential} or \gls{traumatic} amputation severely reduces a person's ability to interact with- / understand the world, express themselves and perform simple day-to-day tasks.
In order to alleviate some of the drawbacks of missing a limb, amputees are often able to aquire a prosthetic replacement of their lost limb.
% HISTORY
The development of prosthetics dates back thousands of years \cite{Kevin2014}.
Early prosthetics would often be fasioned from metal or wooden components, with passive joints enabeling crude movements of the device. 
There has been a large historical development in the area of controllable prosthetics, actuated mechanisms controlled by cable-transmission of tension through upper body would allow simple control of an end gripper.
More modern prosthetics makes use of electronic actuation to simulate the movements of real limbs.
These prostetics 
The modern prosthetic devices tries to imitate the movements of the lost limb, through muscle-activated interfaces used to imitate movements of the real limb. 
% prostetic examples
The upper-body limbs is much more complicated in design and function than the lower-body limbs.
Lower-body prostetics are often mechanical, designed to provide a stable platform for walking \cite{mechanicallegs}.
Imitated functionality of the upper-body limb is much more difficult to design, and requires robotic parts that can actively move and conform based on the user's intension and the interactable objects in the environment.  
In the case of hand prosthetics, it is crucial that the prosthetic allows the user to perform simple day-to-day tasks.
%in order to alleviate some of the stress caused to the non-amputated hand through overusage.
%TODO: Refrence here
%TODO: Get a cool image of a prosthetics user
%\begin{figure}[h]
%\begin{center}
%\includegraphics[width=0.8\textwidth]{example-image-a}
%\caption{Example figure text}
%\label{fig:template}
%\end{center}
%\end{figure}

% MY FOCUS
Statistics show that in 2017, $\sim 57.7 \text{million}$ people were living with amputation due to a \gls{traumatic} occourance \cite{McDonald2020}.
%The statistics indicate that prosthetics research and development would be highly beneficial worldwide.
This indicates that prosthetics research and development would be highly beneficial worldwide.
Research in prosthetics devices and their integration with the human body is a crucial part of increasing amputee social functionality and increase their day-to-day functionality.
The development of this thesis is dedicated to the constantly growing need for prosthetics reserach and development, with the purpose of increasing the useability and controllability of robotic prosthetics devices.
The focus area of this thesis is centered around robotic upper-limb prosthetics due to its complicated interface requirements and its higher need for controllability.

\end{document}
